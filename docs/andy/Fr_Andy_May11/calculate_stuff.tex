%2multibyte Version: 5.50.0.2960 CodePage: 65001

\documentclass{article}
%%%%%%%%%%%%%%%%%%%%%%%%%%%%%%%%%%%%%%%%%%%%%%%%%%%%%%%%%%%%%%%%%%%%%%%%%%%%%%%%%%%%%%%%%%%%%%%%%%%%%%%%%%%%%%%%%%%%%%%%%%%%%%%%%%%%%%%%%%%%%%%%%%%%%%%%%%%%%%%%%%%%%%%%%%%%%%%%%%%%%%%%%%%%%%%%%%%%%%%%%%%%%%%%%%%%%%%%%%%%%%%%%%%%%%%%%%%%%%%%%%%%%%%%%%%%
%TCIDATA{OutputFilter=LATEX.DLL}
%TCIDATA{Version=5.50.0.2960}
%TCIDATA{Codepage=65001}
%TCIDATA{<META NAME="SaveForMode" CONTENT="1">}
%TCIDATA{BibliographyScheme=Manual}
%TCIDATA{Created=Monday, May 25, 2020 14:30:56}
%TCIDATA{LastRevised=Thursday, May 28, 2020 19:02:38}
%TCIDATA{<META NAME="GraphicsSave" CONTENT="32">}
%TCIDATA{<META NAME="DocumentShell" CONTENT="Standard LaTeX\Blank - Standard LaTeX Article">}
%TCIDATA{CSTFile=40 LaTeX article.cst}

\newtheorem{theorem}{Theorem}
\newtheorem{acknowledgement}[theorem]{Acknowledgement}
\newtheorem{algorithm}[theorem]{Algorithm}
\newtheorem{axiom}[theorem]{Axiom}
\newtheorem{case}[theorem]{Case}
\newtheorem{claim}[theorem]{Claim}
\newtheorem{conclusion}[theorem]{Conclusion}
\newtheorem{condition}[theorem]{Condition}
\newtheorem{conjecture}[theorem]{Conjecture}
\newtheorem{corollary}[theorem]{Corollary}
\newtheorem{criterion}[theorem]{Criterion}
\newtheorem{definition}[theorem]{Definition}
\newtheorem{example}[theorem]{Example}
\newtheorem{exercise}[theorem]{Exercise}
\newtheorem{lemma}[theorem]{Lemma}
\newtheorem{notation}[theorem]{Notation}
\newtheorem{problem}[theorem]{Problem}
\newtheorem{proposition}[theorem]{Proposition}
\newtheorem{remark}[theorem]{Remark}
\newtheorem{solution}[theorem]{Solution}
\newtheorem{summary}[theorem]{Summary}
\newenvironment{proof}[1][Proof]{\noindent\textbf{#1.} }{\ \rule{0.5em}{0.5em}}
%\input{tcilatex}f\texttt{}

\begin{document}


\bigskip 

\bigskip 

\bigskip 

\bigskip 

$k\equiv \frac{\left( n-1\right) \sigma _{\mu }^{2}\left( n\right) }{n\sigma
^{2}}$

\bigskip 

From the file Calculated values 2005 the estimate using data up to 2005 is

\bigskip 

$\frac{\sigma ^{2}}{b^{2}}=10249$

\bigskip 

From the file "temp for Larry GLS.doc" the estimate of variance for the
regional model is $\frac{\sigma _{\mu }^{2}}{b^{2}}=4333160$

\bigskip 

If I believed these two numbers I get $k\equiv \frac{\left( n-1\right)
\sigma _{\mu }^{2}\left( n\right) }{n\sigma ^{2}}=1268.\,\allowbreak 365\,694
$

\bigskip 

\bigskip 

\begin{equation}
\widehat{var\left( b_{0i}\right) }\equiv \left( \frac{1}{n}%
\sum_{i}b_{0i}^{2}\right) -\left( \frac{1}{n}\sum_{i}b_{0i}^{{}}\right) ^{2}.
\label{eq var b_0,i}
\end{equation}

\bigskip 

The b\_i for the three regions Bric, Eu, and Other are (taken from the GLS
estimation "temp for Larry GLS estimation"

$b_{B}=4296$

$b_{E}=2745$

$b_{0}=4186$

$\bigskip $

The estimate from "calculated values 2005" of the intercept in the aggregate
model is $\bigskip $

$B_{0}=-5349$

The missing region is Canada and US, call this $b_{N}$

\bigskip 

I have $-5349=\frac{4296+2745+4186+x}{4}$, Solution is: $-32\,623$

\bigskip 

\bigskip 

\begin{enumerate}
\item Indeed, I was using an old estimate,~$\frac{\sigma ^{2}}{b^{2}}%
=1.61e+12$, not the one in the Feb 25 file,~$\frac{\sigma ^{2}}{b^{2}}=10249$%
. Your GLS estimate is $\frac{\sigma _{\mu }^{2}}{nb^{2}}=4333159$. So the
estimate of $k$ is 
\[
k=\left( n-1\right) \frac{\frac{\sigma _{\mu }^{2}}{nb^{2}}}{\frac{\sigma
^{2}}{b^{2}}}=3\frac{4333159}{10249}=\allowbreak 425.\allowbreak 8
\]%
With this correction, the estimate of $k$ goes from being essentially 0 to
426. It is possible that this number is correct. It means that absent trade,
the quantity-based policy is not even in the running in the competition with
taxes. This could well be the case.~ I had it in my mind that $k$ should be
a number like 1.... but now I don't know why I thought that. There is no
apparent upper bound on $k.$ The magnitude of $k$ still worries me, but you
have assured me that you used the same units of emissions for both the
regional and the aggregate models... That would be an easy source of error.

\item You wrote "I just tried to calculate Eq. 25 of the March 5 manuscript,
but realized this equation calls for the inverse of the Omega matrix.~
However, I think the determinant of this matrix is zero, so its inverse does
not exist." I think that you might be confusing the nxn matrix $\tilde{\Omega%
}$ defined in the proof of Remark 1 (page 40 of the March 5 document) with
the (n-1)x(n-1) matrix $\Omega $ defined above equation 25 on page 29. The
former is indeed singular (as I note on page 40), but the inverse of the
latter is 
\[
\Omega ^{-1}=\left( 
\begin{array}{ccc}
4-1 & -1 & -1 \\ 
-1 & 4-1 & -1 \\ 
-1 & -1 & 4-1%
\end{array}%
\right) ^{-1}=\left( 
\begin{array}{ccc}
\frac{1}{2} & \frac{1}{4} & \frac{1}{4} \\ 
\frac{1}{4} & \frac{1}{2} & \frac{1}{4} \\ 
\frac{1}{4} & \frac{1}{4} & \frac{1}{2}%
\end{array}%
\right) 
\]%
My main reason for wanting to calculate 25 was as a consistency check
between your GLS estimation and my algebra. I am pretty confident that my
algebra is correct, and you appear to be pretty confident that your
estimation is correct. Given the amount of time that you have sunk into this
project, I think that we can dispense with calculating 25.

\item The region fixed effect in the file "temp for larry GLS" is actually $%
\frac{b_{0i}-B_{0}}{b}$. (This incorrect labelling is something that you
inherited from a mistake that I made early on.) The three numbers for
regions B,E,O (BRIC, Europe, Other) are $4296.5$, $2745$, and $4185$ (from
the file temp for larry GLS)
\end{enumerate}

\bigskip 

\bigskip 

\end{document}
